\documentclass{beamer}

\usepackage{luxbeamer}

\title{LuXBeamer}
\subtitle{Ein Minimalbeispiel}
\author{Ervin Mazlagi\'c}

\begin{document}

\maketitle


\section{Programm}
\begin{frame}
\tableofcontents
\end{frame}

\section{Slide-Grundlagen}

\subsection{Blöcke}

%%%%% ein Slide
\begin{frame}
	\frametitle{Blöcke \hfill \footnotesize{LuXeria}}
	\framesubtitle{Standardblöcke im Vergleich \hfill \tiny{OpenSource --- Open Mind!}}
	
	\begin{block}{Regulärer Block}
		Das ist ein regulärer Block.
	\end{block}

	\begin{alertblock}{Alarmblock}
		Das ist ein Alarmblock.
	\end{alertblock}

	\begin{exampleblock}{Beispielblock}
		Das ist ein Beispielblock.
	\end{exampleblock}
\end{frame}

\subsection{Kolonnen}

\begin{frame}
	\frametitle{Slides splaten \hfill \footnotesize{LuXeria}}
	\framesubtitle{Vertikale Trennung \hfill \tiny{OpenSource}}

	\begin{columns}
		\begin{column}{5cm}
			\begin{block}{Links}
				Linke Kolonne
			\end{block}
		\end{column}
		\begin{column}{5cm}
			\begin{block}{Rechts}
				Rechte Kolonne
			\end{block}
		\end{column}
	\end{columns}
\end{frame}



\end{document}
