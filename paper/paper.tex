\documentclass[a4paper,
               10pt,
               fleqn]{article}

\author{Ervin Mazlagic}
\title{Paper Manual}

\usepackage{luxpaper}
\usepackage{luxtitle}
 
\begin{document}
\luxtitle{Papers}
         {LuXPaper Manual}
         {Ervin Mazlagi\'c}
         {Adligenswil}
         {2012}

\tableofcontents
\newpage

\section{Einleitung}
Das Package ``luxpaper'' ist die Wahl, wenn es darum geht 
eine Dokumentation oder Arbeit zu schreiben.
In diesem Package sind alle gängigen Deklarationen vorhanden
wie etwa die Spracheinstellung, Seiteneinzüge, Kopf- und
Fusszeilendeklarationen etc.
Spezielle Packages die über den Alltagsgebrauch hinaus gehen
sind ebenfalls bereits enthalten (TikZ, PGF-PIE, CircuiTikZ etc.).
Das lstlistings-Package\footnote{
    Das Package ``lstlistings'' ist jenes, welches benutzt wird um
    SourceCode in ein Dokument einzubinden.}
wird separat behandelt in diesem Manual (siehe Kapitel 
%\ref{sec:lstlisting}).

Dieses Manual soll das Package in seiner Gesamtheit dokumentieren
und Anweisungen für den Gebrauch liefern.

\section{Regeln \& Konventionen}
In der Welt der Coder sind Spielregeln und Konventionen nichts 
besonderes. Jedoch liefern diese den Stoff zum Cyber-Krieg
(Stihwort K\&R vs. \dots). 

Hier sollen wirklich nur funktional relevante Konventionen
aufgestellt und keine Style-Diktatur betrieben werden.

\subsection{Header}
Das Main File eines Dokumentes für ``LuXeria Papers'' muss den 
folgenden Header aufweisen.

\begin{lstlisting}
\documentclass[a4paper,             % Papierformat      DIN A4
               10pt,                % Schriftgrösse     10pt
               fleqn]               % Formelausrichtung fleqn
               {article}            % Klasse            article

\author{Vorname Nachname}
\title{Titel der Arbeit}

\usepackage{luxpaper}
\usepackage{luxtitle}

\begin{document}
    
    \luxtitle{Klasse}               % Art der Arbeit (Paper, Report...)
             {Titel}                % Titel der Arbeit
             {Autor}                % Verfasser der Arbeit
             {Ort}                  % Ort (Adligenswil)
             {Jahr}                 % Jahr des Verfassens

    \tableofcontents                % Inhaltsverzeichnis
    \newpage                        % Seitenumbruch

    % Inhalt folt an dieser Stelle bis zum \end{document}

\end{document}
\end{lstlisting}

\noindent
Zwischen dem \lstinline$\usepackage{luxtitle}$ in Zeile 10 und 
den \lstinline$\begin{document}$ in Zeile 12 können weitere
Packages geladen werden. Wichtig ist, das die zwei hier genannten
Packages als erste einbezogen werden.

\section{Package}
Das ``luxpaper''-Package muss bei einem neuen Dokument angepasst
werden. Nämlich an folgenden Stellen:

\begin{itemize}
    \item Kopfzeile Links \hfill (Zeile 18)
    \item Fusszeile Links \hfill (Zeile 22)
\end{itemize}

\noindent
Dort ist jeweils der korrekte Inhalt einzufügen (Titel der Arbeit und der
Name des Autors).

\lstinputlisting[firstline=109, lastline=134]{luxpaper.sty}

\subsection{Source}
Die Source zum Package ist auf \url{www.gihub.com/luxeria/luxtex}
hinterlegt.

\section{Titelseite}
Das Titelseiten-Package ``luxtitle'' ermöglicht ein genormetes 
Titelblatt für die LuXeria. Dieses stellt ein LaTeX-Commando 
zur Verfügung mit 5 Parametern.

\begin{lstlisting}
\luxtitle{Klasse}   % Klasse der Arbeit (Paper, Report...)
         {Titel}    % Titel der Arbeit
         {Autor}    % Autor der Arbeit
         {Ort}      % Ort des Verfassens
         {Jahr}     % Jahr des Verfassens
\end{lstlisting}

\end{document}
